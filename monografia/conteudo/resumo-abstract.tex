% !TeX root=../tese.tex
%("dica" para o editor de texto: este arquivo é parte de um documento maior)
% para saber mais: https://tex.stackexchange.com/q/78101/183146
% O resumo é obrigatório, em português e inglês. Este comando também gera
% automaticamente a referência para o próprio documento, conforme as normas
% sugeridas da USP
\begin{resumo}{port}
  Este trabalho de \textbr{conclusão} de curso trata de um dos mais fascinantes e úteis conceitos em ciência da computação: funções de hash e tabelas de hash. O texto está organizado em três partes principais:

\begin{itemize}
  \item Funções de Hash
  \item Tabelas de Hash
  \item Aplicações
\end{itemize}

Funções de hash é uma ideia importante em ciência da computação e vai muito além de seu uso em tabelas de hash. Nesse texto são descritas algumas das ideias que Donald Knuth apresentou em seu livro inspirador, \textit{The Art of Computer programming, Vol. 3} \citep{TAOCP3}. Estimamos a qualidade de funções de hash através de algumas métricas conhecidas.

Tabelas de hash é uma das mais usadas estruturas de dados em programação. Indicamos os componentes de tabelas de hash em que funções de hash têm um papel primordial. Descrevemos várias das implementações clássicas dessa estrutura; cada uma apropriada para um determinado cenário.

Por fim, descrevemos algumas aplicações de funções e tabelas de hash em problemas recorrentes em ciência da computação. Entre as aplicações está o algoritmo \textit{Rabin-Karp} para busca de padrão em textos que utiliza hashing e um algoritmo eficiente para identificar isomorfismo em árvores utilizando funções de hash.

Espero que esse trabalho seja tão divertido de ler quanto foi para escrever!

Obs: O idioma escolhido para o trabalho foi o inglês devido a muitos termos que se referem a hashing estarem nesse idoma. 

\end{resumo}

% O resumo é obrigatório, em português e inglês. Este comando também gera
% automaticamente a referência para o próprio documento, conforme as normas
% sugeridas da USP
\begin{resumo}{eng}
  
This text deals with one of the most fascinating and useful concepts in Computer Science, which are hash functions and hash tables. It is organized in three main topics:

\begin{itemize}
   \item Hash functions
   \item Hash tables
   \item Applications
\end{itemize}

Hash functions is a key tool in Computer Science, its applications goes far beyond its use in hash tables. In this text it is presented some of the ideas Donald Knuth described in his inspiring book, \textit{The Art of Computer programming, Vol. 3} \citep{TAOCP3}, and we apply some metrics in order to estimate the quality of a hash function.

Hash tables is one of the most used data structures in computer programming. We present the constituents parts of a hash table, in which hash functions have a prominent role, and show some of the classic implementations of this data structure; each one particularly useful in a specific scenario.

Finally, we describe some applications of hash functions and hash tables in every day computer science problems. Among the algorithms shown there are \textit{Rabin-Karp}, a string search algorithm that uses hashing and a solution to identify isomorphism on trees using hashing functions.

I hope this is as fun to read for you as it was for me to write!
\\

\end{resumo}
