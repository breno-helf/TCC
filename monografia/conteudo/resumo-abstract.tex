% !TeX root=../tese.tex
%("dica" para o editor de texto: este arquivo é parte de um documento maior)
% para saber mais: https://tex.stackexchange.com/q/78101/183146
% O resumo é obrigatório, em português e inglês. Este comando também gera
% automaticamente a referência para o próprio documento, conforme as normas
% sugeridas da USP
\begin{resumo}{port}
  Este trabalho de \textbr{conclusão} de curso é sobre um dos mais fascinantes e úteis conceitos em ciência da computação, funções de hash e tabelas de hash. O texto está organizado em três tópicos principais:

\begin{itemize}
  \item Funções de Hash
  \item Tabelas de Hash
  \item Aplicações
\end{itemize}

Funções de Hash é uma ideia importante em ciência da computação e vai muito além de seu uso em tabelas de hash. Durante o trabalho é apresentada algumas ideias que Donald Knuth apresentou em seu livro, \textit{The Art of Computer programming, Vol. 3} \citep{TAOCP3}, e usamos algumas metricas para estimar a qualidade de funções de hash.

Tabelas de hash é uma das mais usadas estruturas de dados em programação. É apresentado as partes essenciais de uma tabela de hash, em que funções de hash são um ingrediente essencial, e mostramos algumas das implementações clássicas dessa estrutura de dados.

Por fim, é descrito algumas aplicações de funções e tabelas de hash em problemas recorrentes em ciência da computação. Entre os algorítmos mostrados estão \textit{Rabin-Karp}, um algortimo de busca de padrão em textos que utiliza hashing e uma solução para identificar isomorfismo em árvores utilizando funções de hash.

Espero que esse trabalho seja tão divertido de ler quanto foi para escrever!

Obs: O idioma escolhido para o trabalho foi o inglês devido a muitos termos que se referem a hashing estarem nesse idoma. 

\end{resumo}

% O resumo é obrigatório, em português e inglês. Este comando também gera
% automaticamente a referência para o próprio documento, conforme as normas
% sugeridas da USP
\begin{resumo}{eng}
  
This text is about one of the most fascinating and useful concepts in Computer Science, which are hash functions and hash tables. It is organized in three main topics:

\begin{itemize}
   \item Hash functions
   \item Hash tables
   \item Applications
\end{itemize}

Hash functions are an important idea in Computer Science, that goes far beyond its use in hash tables. During the text it is presented some of the ideas Donald Knuth presented on his book, \textit{The Art of Computer programming, Vol. 3} \citep{TAOCP3}, and we apply some metrics in order to estimate the quality of a hash function.

Hash tables is one of the most used data structures in computer programming. We present the constituents parts of a hash table, which hash functions are a key ingredient, and show some of the classic implementations of this data structure. 

Finally, we describe some applications of hash functions and hash tables in every day computer science problems. Among the algorithms show there are \textit{Rabin-Karp}, a string search algorithm that uses hashing and a solution to identify isomorphism on trees using hashing functions.

I hope this is as fun to read for you as it was for me to write!
\\

\end{resumo}
