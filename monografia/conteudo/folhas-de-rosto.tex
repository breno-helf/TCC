%!TeX root=../tese.tex
%("dica" para o editor de texto: este arquivo é parte de um documento maior)
% para saber mais: https://tex.stackexchange.com/q/78101/183146

%%%%%%%%%%%%%%%%%%%%%%%%%%%%%%%%%%%%%%%%%%%%%%%%%%%%%%%%%%%%%%%%%%%%%%%%%%%%%%%%
%%%%%%%%%%%%%%%%%%%%%%%%%%%%% METADADOS DA TESE %%%%%%%%%%%%%%%%%%%%%%%%%%%%%%%%
%%%%%%%%%%%%%%%%%%%%%%%%%%%%%%%%%%%%%%%%%%%%%%%%%%%%%%%%%%%%%%%%%%%%%%%%%%%%%%%%

% Define o texto da capa e da referência que vai na página do resumo;
% "masc" ou "fem" definem se serão usadas palavras no masculino ou feminino
% (Mestre/Mestra, Doutor/Doutora, candidato/candidata). O segundo parâmetro
% é opcional e determina que se trata de exame de qualificação.
%\mestrado[fem]
%\mestrado[fem][quali]
%\doutorado[masc]
%\doutorado[masc][quali]
\tcc[masc]

% Se "\title" está em inglês, você pode definir o título em português aqui
\tituloport{Funções e Tables de Hash}

% Se "\title" está em português, você pode definir o título em inglês aqui
\tituloeng{Hash Functions and Hash Tables}

% Para TCCs, este comando define o supervisor
%\orientador[fem]{Profª. Drª. Fulana de Tal}

% Se isto não for definido, "\orientador" é utilizado no lugar
\orientadoreng{Jose Coelho de Pina Junior}

% Se não houver, remova
%\coorientador[masc]{Prof. Dr. Ciclano}

% Se isto não for definido, "\coorientador" é utilizado no lugar
%\coorientadoreng{Prof. Dr. Ciclano}

%\programa{Ciência da Computação}

% Se isto não for definido, "\programa" é utilizado no lugar
\programaeng{Computer Science}

% Se não houver, remova
%\apoio{Durante o desenvolvimento deste trabalho o autor recebeu auxílio
%financeiro da XXXX}

% Se isto não for definido, "\apoio" é utilizado no lugar
%\apoioeng{During this work, the author was supported by XXX}

\localdefesa{São Paulo}

%\datadefesa{10 de Dezembro de 2019}

% Se isto não for definido, "\datadefesa" é utilizado no lugar
\datadefesaeng{December 10th, 2019}

% Necessário para criar a referência do documento que aparece
% na página do resumo
\ano{2019}

%\banca{
%  \begin{itemize}
%    \item Profª. Drª. Nome Completo (orientadora) - IME-USP [sem ponto final]
%    \item Prof. Dr. Nome Completo - IME-USP [sem ponto final]
%    \item Prof. Dr. Nome Completo - IMPA [sem ponto final]
%  \end{itemize}
%}

% Se isto não for definido, "\banca" é utilizado no lugar
%\bancaeng{
%  \begin{itemize}
%    \item Prof. Dr. Nome Completo (advisor) - IME-USP [sem ponto final]
%    \item Prof. Dr. Nome Completo - IME-USP [sem ponto final]
%    \item Prof. Dr. Nome Completo - IMPA [sem ponto final]
%  \end{itemize}
%}

% Palavras-chave separadas por ponto e finalizadas também com ponto.
\palavraschave{Hash. Hashing. Table. Function.}

\keywords{Hash. Hashing. Table. Function.}

% Se quiser estabelecer regras diferentes, converse com seu
% orientador
%\direitos{Autorizo a reprodução e divulgação total ou parcial
%deste trabalho, por qualquer meio convencional ou
%eletrônico, para fins de estudo e pesquisa, desde que
%citada a fonte.}

% Isto deve ser preparado em conjunto com o bibliotecário
%\fichacatalografica{
% nome do autor, título, etc.
%}

%%%%%%%%%%%%%%%%%%%%%%%%%%% CAPA E FOLHAS DE ROSTO %%%%%%%%%%%%%%%%%%%%%%%%%%%%%

% Embora as páginas iniciais *pareçam* não ter numeração, a numeração existe,
% só não é impressa. O comando \mainmatter (mais abaixo) reinicia a contagem
% de páginas e elas passam a ser impressas. Isso significa que existem duas
% páginas com o número "1": a capa e a página do primeiro capítulo. O pacote
% hyperref não lida bem com essa situação. Assim, vamos desabilitar hyperlinks
% para números de páginas no início do documento e reabilitar mais adiante.
\hypersetup{pageanchor=false}

% A capa; o parâmetro pode ser "port" ou "eng" para definir a língua
%\capaime[port]
\capaime[eng]

% Se você não quiser usar a capa padrão, você pode criar uma outra
% capa manualmente ou em um programa diferente. No segundo caso, é só
% importar a capa como uma página adicional usando o pacote pdfpages.
%\includepdf{./arquivo_da_capa.pdf}

% A página de rosto da versão para depósito (ou seja, a versão final
% antes da defesa) deve ser diferente da página de rosto da versão
% definitiva (ou seja, a versão final após a incorporação das sugestões
% da banca). Os parâmetros podem ser "port/eng" para a língua e
% "provisoria/definitiva" para o tipo de página de rosto.
% Observe que TCCs não têm página de rosto; nesse caso, desabilite
% todas as opções.
%\pagrostoime[port]{definitiva}
%\pagrostoime[port]{provisoria}
%\pagrostoime[eng]{definitiva}
%\pagrostoime[eng]{provisoria}

%%%%%%%%%%%%%%%%%%%% DEDICATÓRIA, RESUMO, AGRADECIMENTOS %%%%%%%%%%%%%%%%%%%%%%%

% A definição deste ambiente está no pacote imeusp-capa.sty; se você não
% carregar esse pacote, precisa cuidar desta página manualmente.

% Após a capa e as páginas de rosto, começamos a numerar as páginas; com isso,
% podemos também reabilitar links para números de páginas no pacote hyperref.
% Isso porque, embora contagem de páginas aqui começe em 1 e no primeiro
% capítulo também, o fato de uma numeração usar algarismos romanos e a outra
% algarismos arábicos é suficiente para evitar problemas.
\pagenumbering{roman}
\hypersetup{pageanchor=true}

% !TeX root=../tese.tex
%("dica" para o editor de texto: este arquivo é parte de um documento maior)
% para saber mais: https://tex.stackexchange.com/q/78101/183146
% O resumo é obrigatório, em português e inglês. Este comando também gera
% automaticamente a referência para o próprio documento, conforme as normas
% sugeridas da USP
\begin{resumo}{port}
  Este trabalho de \textbr{conclusão} de curso trata de um dos mais fascinantes e úteis conceitos em ciência da computação: funções de hash e tabelas de hash. O texto está organizado em três partes principais:

\begin{itemize}
  \item Funções de Hash
  \item Tabelas de Hash
  \item Aplicações
\end{itemize}

Funções de hash é uma ideia importante em ciência da computação e vai muito além de seu uso em tabelas de hash. Nesse texto são descritas algumas das ideias que Donald Knuth apresentou em seu livro inspirador, \textit{The Art of Computer programming, Vol. 3} \citep{TAOCP3}. Estimamos a qualidade de funções de hash através de algumas métricas conhecidas.

Tabelas de hash é uma das mais usadas estruturas de dados em programação. Indicamos os componentes de tabelas de hash em que funções de hash têm um papel primordial. Descrevemos várias das implementações clássicas dessa estrutura; cada uma apropriada para um determinado cenário.

Por fim, descrevemos algumas aplicações de funções e tabelas de hash em problemas recorrentes em ciência da computação. Entre as aplicações está o algoritmo \textit{Rabin-Karp} para busca de padrão em textos que utiliza hashing e um algoritmo eficiente para identificar isomorfismo em árvores utilizando funções de hash.

Espero que esse trabalho seja tão divertido de ler quanto foi para escrever!

Obs: O idioma escolhido para o trabalho foi o inglês devido a muitos termos que se referem a hashing estarem nesse idoma. 

\end{resumo}

% O resumo é obrigatório, em português e inglês. Este comando também gera
% automaticamente a referência para o próprio documento, conforme as normas
% sugeridas da USP
\begin{resumo}{eng}
  
This text deals with one of the most fascinating and useful concepts in Computer Science, which are hash functions and hash tables. It is organized in three main topics:

\begin{itemize}
   \item Hash functions
   \item Hash tables
   \item Applications
\end{itemize}

Hash functions is a key tool in Computer Science, its applications goes far beyond its use in hash tables. In this text it is presented some of the ideas Donald Knuth described in his inspiring book, \textit{The Art of Computer programming, Vol. 3} \citep{TAOCP3}, and we apply some metrics in order to estimate the quality of a hash function.

Hash tables is one of the most used data structures in computer programming. We present the constituents parts of a hash table, in which hash functions have a prominent role, and show some of the classic implementations of this data structure; each one particularly useful in a specific scenario.

Finally, we describe some applications of hash functions and hash tables in every day computer science problems. Among the algorithms shown there are \textit{Rabin-Karp}, a string search algorithm that uses hashing and a solution to identify isomorphism on trees using hashing functions.

I hope this is as fun to read for you as it was for me to write!
\\

\end{resumo}



%%%%%%%%%%%%%%%%%%%%%%%%%%% LISTAS DE FIGURAS ETC. %%%%%%%%%%%%%%%%%%%%%%%%%%%%%

% Como as listas que se seguem podem não incluir uma quebra de página
% obrigatória, inserimos uma quebra manualmente aqui.
\makeatletter
\if@openright\cleardoublepage\else\clearpage\fi
\makeatother

% Todas as listas são opcionais; Usando "\chapter*" elas não são incluídas
% no sumário. As listas geradas automaticamente também não são incluídas
% por conta das opções "notlot" e "notlof" que usamos mais acima.

% Normalmente, "\chapter*" faz o novo capítulo iniciar em uma nova página, e as
% listas geradas automaticamente também por padrão ficam em páginas separadas.
% Como cada uma destas listas é muito curta, não faz muito sentido fazer isso
% aqui, então usamos este comando para desabilitar essas quebras de página.
% Se você preferir, comente as linhas com esse comando e des-comente as linhas
% sem ele para criar as listas em páginas separadas. Observe que você também
% pode inserir quebras de página manualmente (com \clearpage, veja o exemplo
% mais abaixo).
\newcommand\disablenewpage[1]{{\let\clearpage\par\let\cleardoublepage\par #1}}

% Nestas listas, é melhor usar "raggedbottom" (veja basics.tex). Colocamos
% a opção correspondente e as listas dentro de um par de chaves para ativar
% raggedbottom apenas temporariamente.
{
\raggedbottom

%%%%% Listas criadas manualmente

%\chapter*{Lista de Abreviaturas}
%\disablenewpage{\chapter*{Lista de Abreviaturas}}

%\begin{tabular}{rl}
%         CFT         & Transformada contínua de Fourier (\emph{Continuous Fourier Transform})\\
%         DFT         & Transformada discreta de Fourier (\emph{Discrete Fourier Transform})\\
%        EIIP         & Potencial de interação elétron-íon (\emph{Electron-Ion Interaction Potentials})\\
%        STFT         & Transformada de Fourier de tempo reduzido (\emph{Short-Time Fourier Transform})\\
%	ABNT         & Associação Brasileira de Normas Técnicas\\
%	URL          & Localizador Uniforme de Recursos (\emph{Uniform Resource Locator})\\
%	IME          & Instituto de Matemática e Estatística\\
%	USP          & Universidade de São Paulo
%\end{tabular}

%\chapter*{Lista de Símbolos}
%\disablenewpage{\chapter*{Lista de Símbolos}}

%\begin{tabular}{rl}
%       $\omega$    & Frequência angular\\
%      $\psi$      & Função de análise \emph{wavelet}\\
%     $\Psi$      & Transformada de Fourier de $\psi$\\
%\end{tabular}

% Quebra de página manual
\clearpage

%%%%% Listas criadas automaticamente

%\listoffigures
%\disablenewpage{\listoffigures}

%\listoftables
%\disablenewpage{\listoftables}

% Esta lista é criada "automaticamente" pela package float quando
% definimos o novo tipo de float "program" (em utils.tex)
%\listof{program}{\programlistname}
%\disablenewpage{\listof{program}{\programlistname}}

% Sumário (obrigatório)
\tableofcontents

} % Final de "raggedbottom"

% Referências indiretas ("x", veja "y") para o índice remissivo (opcionais,
% pois o índice é opcional). É comum colocar esses itens no final do documento,
% junto com o comando \printindex, mas em alguns casos isso torna necessário
% executar texindy (ou makeindex) mais de uma vez, então colocar aqui é melhor.
%\index{Inglês|see{Língua estrangeira}}
%\index{Figuras|see{Floats}}
%\index{Tabelas|see{Floats}}
%\index{Código-fonte|see{Floats}}
%\index{Subcaptions|see{Subfiguras}}
%\index{Sublegendas|see{Subfiguras}}
%\index{Equações|see{Modo Matemático}}
%\index{Fórmulas|see{Modo Matemático}}
%\index{Rodapé, notas|see{Notas de rodapé}}
%\index{Captions|see{Legendas}}
%\index{Versão original|see{Tese/Dissertação, versões}}
%\index{Versão corrigida|see{Tese/Dissertação, versões}}
%\index{Palavras estrangeiras|see{Língua estrangeira}}
%\index{Floats!Algoritmo|see{Floats, Ordem}}
