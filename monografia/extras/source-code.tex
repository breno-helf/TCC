% Para formatar código-fonte (ex. em Java). listings funciona bem mas
% tem algumas limitações (https://tex.stackexchange.com/a/153915 ).
% Se isso for um problema, a package minted pode oferecer resultados
% melhores; a desvantagem é que ela depende de um programa externo,
% o pygments (escrito em python).
%
% listings também não tem suporte específico a pseudo-código, mas
% incluímos uma configuração para isso que deve ser suficiente.
% Caso contrário, há diversas packages específicas para a criação
% de pseudocódigo:
%
%  * a mais comum é algorithmicx ("\usepackage{algpseudocode}");
%  * algorithm2e é bastante flexível, mas um tanto complexa;
%  * clrscode3e foi usada no livro "Introduction to Algorithms"
%    de Cormen, Leiserson, Rivert e Stein.
%  * pseudocode foi usada no livro "Combinatorial Algorithms"
%    de Kreher e Stinson.

\usepackage{xcolor, listings}
\usepackage{lstautogobble}
% Carrega a "linguagem" pseudocode para listings
\appto{\lstaspectfiles}{,lstpseudocode.sty}
\appto{\lstlanguagefiles}{,lstpseudocode.sty}
\dowithsubdir{extras/}{\lstloadaspects{simulatex,invisibledelims,pseudocode}}
\dowithsubdir{extras/}{\lstloadlanguages{[base]pseudocode,[english]pseudocode,[brazil]pseudocode}}

% O pacote listings não lida bem com acentos! No caso dos caracteres acentuados
% usados em português é possível contornar o problema com a tabela abaixo.
% From https://en.wikibooks.org/wiki/LaTeX/Source_Code_Listings#Encoding_issue
%\lstset{literate=
%  {á}{{\'a}}1 {é}{{\'e}}1 {í}{{\'i}}1 {ó}{{\'o}}1 {ú}{{\'u}}1
%  {Á}{{\'A}}1 {É}{{\'E}}1 {Í}{{\'I}}1 {Ó}{{\'O}}1 {Ú}{{\'U}}1
%  {à}{{\`a}}1 {è}{{\`e}}1 {ì}{{\`i}}1 {ò}{{\`o}}1 {ù}{{\`u}}1
%  {À}{{\`A}}1 {È}{{\'E}}1 {Ì}{{\`I}}1 {Ò}{{\`O}}1 {Ù}{{\`U}}1
%  {ä}{{\"a}}1 {ë}{{\"e}}1 {ï}{{\"i}}1 {ö}{{\"o}}1 {ü}{{\"u}}1
%  {Ä}{{\"A}}1 {Ë}{{\"E}}1 {Ï}{{\"I}}1 {Ö}{{\"O}}1 {Ü}{{\"U}}1
%  {â}{{\^a}}1 {ê}{{\^e}}1 {î}{{\^i}}1 {ô}{{\^o}}1 {û}{{\^u}}1
%  {Â}{{\^A}}1 {Ê}{{\^E}}1 {Î}{{\^I}}1 {Ô}{{\^O}}1 {Û}{{\^U}}1
%  {Ã}{{\~A}}1 {ã}{{\~a}}1 {Õ}{{\~O}}1 {õ}{{\~o}}1
%  {œ}{{\oe}}1 {Œ}{{\OE}}1 {æ}{{\ae}}1 {Æ}{{\AE}}1 {ß}{{\ss}}1
%  {ű}{{\H{u}}}1 {Ű}{{\H{U}}}1 {ő}{{\H{o}}}1 {Ő}{{\H{O}}}1
%  {ç}{{\c c}}1 {Ç}{{\c C}}1 {ø}{{\o}}1 {å}{{\r a}}1 {Å}{{\r A}}1
%  {€}{{\euro}}1 {£}{{\pounds}}1 {«}{{\guillemotleft}}1
%  {»}{{\guillemotright}}1 {ñ}{{\~n}}1 {Ñ}{{\~N}}1 {¿}{{?`}}1
%}

% Opções default para o pacote listings
% Ref: http://en.wikibooks.org/wiki/LaTeX/Packages/Listings
\lstset{language=C++,
  basicstyle=\footnotesize\ttfamily,
  keywordstyle=\color{blue}\bfseries,
  commentstyle=\color{green},
  stringstyle=\ttfamily\color{red!50!brown},
  showstringspaces=false,
  frame=single,	                % adds a frame around the code
  tabsize=2,	                    % sets default tabsize to 2 spaces
  numbers=left,                   % where to put the line-numbers
  stepnumber=1,                   % the step between two line-numbers. If it's 1 each line will b
  numbersep=5pt,                  % how far the line-numbers are from the code
  showspaces=false,               % show spaces adding particular underscores
  showstringspaces=false,         % underline spaces within strings
  showtabs=false,                 % show tabs within strings adding particular underscores
  framerule=0.6pt,
  captionpos=b,                   % sets the caption-position to bottom
  breaklines=true,                % sets automatic line breaking
  breakatwhitespace=false,        % sets if automatic breaks should only happen at whitespace
  backgroundcolor=\color[rgb]{1.0,1.0,1.0}, % choose the background color.
  xleftmargin=14pt,
  xrightmargin=14pt,
  framexleftmargin=14pt,
  framexrightmargin=14pt
}
\lstset{literate=%
  *{0}{{{\color{red!20!violet}0}}}1
  {1}{{{\color{red!20!violet}1}}}1
  {2}{{{\color{red!20!violet}2}}}1
  {3}{{{\color{red!20!violet}3}}}1
  {4}{{{\color{red!20!violet}4}}}1
  {5}{{{\color{red!20!violet}5}}}1
  {6}{{{\color{red!20!violet}6}}}1
  {7}{{{\color{red!20!violet}7}}}1
  {8}{{{\color{red!20!violet}8}}}1
  {9}{{{\color{red!20!violet}9}}}1
}

%\lstset{
%  columns=[l]fullflexible,            % do not try to align text with proportional fonts
%  basicstyle=\footnotesize\ttfamily,  % the font that is used for the code
%  numbers=left,                       % where to put the line-numbers
%  numberstyle=\footnotesize\ttfamily, % the font that is used for the line-numbers
%  stepnumber=1,                       % the step between two line-numbers. If it's 1 each lin%e will be numbered
%  numbersep=20pt,                     % how far the line-numbers are from the code
%  autogobble,                         % ignore irrelevant indentation
%  commentstyle=\color{Brown}\upshape,
%  stringstyle=\color{black},
%  identifierstyle=\color{DarkBlue},
%  keywordstyle=\color{cyan},
%  showspaces=false,                   % show spaces adding particular underscores
%  showstringspaces=false,             % underline spaces within strings
%  showtabs=false,                     % show tabs within strings adding particular underscores
  %frame=single,                       % adds a frame around the code
%  framerule=0.6pt,
%  tabsize=2,                          % sets default tabsize to 2 spaces
%  captionpos=b,                       % sets the caption-position to bottom
%  breaklines=true,                    % sets automatic line breaking
%  breakatwhitespace=false,            % sets if automatic breaks should only happen at whitespace
%  escapeinside={\%*}{*)},             % if you want to add a comment within your code
%  backgroundcolor=\color[rgb]{1.0,1.0,1.0}, % choose the background color.
%  rulecolor=\color{darkgray},
%  extendedchars=true,
%  inputencoding=utf8,
%  xleftmargin=30pt,
%  xrightmargin=10pt,
%  framexleftmargin=25pt,
%  framexrightmargin=5pt,
%  framesep=5pt,
%}

% Um exemplo de estilo personalizado para listings (tabulações maiores)
%\lstdefinestyle{wider} {
%  tabsize = 4,
%  numbersep=15pt,
%  xleftmargin=25pt,
%  framexleftmargin=20pt,
%}

% Outro exemplo de estilo personalizado para listings (sem cores)
%\lstdefinestyle{nocolor} {
%  commentstyle=\color{darkgray}\upshape,
%  stringstyle=\color{black},
%  identifierstyle=\color{black},
%  keywordstyle=\color{black}\bfseries,
%}

% Um exemplo de definição de linguagem para listings (XML)
%\lstdefinelanguage{XML}{
%  morecomment=[s]{<!--}{-->},
%  morecomment=[s]{<!-- }{ -->},
%  morecomment=[n]{<!--}{-->},
%  morecomment=[n]{<!-- }{ -->},
 % morestring=[b]",
 % morestring=[s]{>}{<},
 % morecomment=[s]{<?}{?>},
 % morekeywords={xmlns,version,type}% list your attributes here
%}

% Estilo padrão para a "linguagem" pseudocode
%\lstdefinestyle{pseudocode}{
%  basicstyle=\rmfamily\small,
%  commentstyle=\itshape,
%  keywordstyle=\bfseries,
%  identifierstyle=\itshape,
%  % as palavras "function" e "procedure"
%  procnamekeystyle=\bfseries\scshape,
%  % funções precedidas por function/procedure ou com \func{}
%  procnamestyle=\ttfamily,
%  specialidentifierstyle=\ttfamily\bfseries,
%}
\lstset{defaultdialect=[english]{C++}}

% A package listings tem seu próprio mecanismo para a criação de
% captions, lista de programas etc. Neste modelo não usamos esses
% recursos (veja mais abaixo), mas utilizamos estes nomes:
\addto\extrasbrazil{%
  \gdef\lstlistlistingname{Lista de Programas}%
  \gdef\lstlistingname{Programa}%
}
\addto\extrasenglish{%
  \gdef\lstlistlistingname{List of Programs}%
  \gdef\lstlistingname{Program}%
}

% Novo tipo de float para programas, possível graças à package floatrow.
% Observe que a lista de floats de cada tipo é criada automaticamente
% pela package floatrow, mas precisamos:
%  1. Definir o nome do comando ("\begin{program}")
%  2. Definir o nome do float em cada língua ("Figura X", "Programa X")
%  3. Definir a extensão do arquivo temporário a ser usada. Pode ser
%     qualquer coisa, desde que não haja repetições. Aqui, usamos "lop";
%     lembre-se que LaTeX já usa várias outras, como "lof", "lot" etc.,
%     então seja cuidadoso na escolha!
%  4. Acrescentar os comandos correspondentes em folhas-de-rosto.tex

\makeatletter
\@ifpackageloaded{floatrow}
  {
    \ltx@IfUndefined{chapter}
        % O novo ambiente se chama "program" ("\begin{program}") e a extensão
        % temporária é "lop"
        {\DeclareNewFloatType{program}{placement=htbp,fileext=lop}}
        {\DeclareNewFloatType{program}{placement=htbp,fileext=lop,within=chapter}}

    % Ajusta ligeiramente o espaçamento do estilo "ruled".
    \DeclareFloatVCode{customrule}{{\kern 0pt\hrule\kern 2.5pt\relax}}
    \floatsetup[program]{style=ruled,precode=customrule}
  }
  {
    % Não temos a package floatrow; vamos assumir que temos a package float.

    % O estilo padrão do novo float a ser criado (veja mais sobre isso na
    % documentação da package float). Para "program" usamos "ruled", mas
    % para outros floats provavelmente é melhor usar o mesmo formato de
    % Figuras e Tables (plain).
    \floatstyle{ruled}

    \ltx@IfUndefined{chapter}
        % O novo ambiente se chama "program" ("\begin{program}") e a extensão
        % temporária é "lop"
        {\newfloat{program}{htbp}{lop}}
        {\newfloat{program}{htbp}{lop}[chapter]}

    % Retorna o estilo dos floats para o padrão
    \floatstyle{plain}
  }
\makeatother

\captionsetup[program]{style=ruled,position=top}

% "Program X / Programa X" e "Lista de Programas / List of Programs"
\floatname{program}{\lstlistingname}
\gdef\programlistname{\lstlistlistingname}

% Se um programa é maior que uma página, ele não pode ser inserido em
% um float. Nesse caso, vamos criar o ambiente "programruledcaption",
% que cria a mesma estrutura visual e os mesmos captions que os floats
% do tipo "program", mas sem ser um float. Para isso, vamos usar recursos
% da package framed (a package tcolorbox poderia ter sido usada também).
%
% Observe que "programruledcaption" funciona *apenas* para os floats do
% tipo "program". Se quiser criar algo similar para outro tipo de float,
% você vai precisar criar um novo comando ("myfloatruledcaption")
% copiando os comandos abaixo e modificando-os conforme necessário.
\newsavebox{\captiontextBox}
\usepackage{framed}
\newenvironment{programruledcaption}[2][]{
  % All spacing measurements were adjusted to visually reproduce
  % the float captions
  \setlength\fboxsep{0pt}

  % topsep means space before AND after
  \setlength\topsep{.28\baselineskip plus .3\baselineskip minus 0pt}

  \vspace{.3\baselineskip} % Some extra top space

  % For whatever reason, the framed package actually calls "\captionof"
  % multiple times, messing up the counter. We need to prevent this,
  % so we put the caption in a box once and reuse the box.

  \savebox{\captiontextBox}{%
    \parbox[b]{\textwidth}{%
      \ifstrempty{#1}
        {\captionof{program}[#2]{#2}}%
        {\captionof{program}[#1]{#2}}%
    }
  }

  \def\fullcaption{
    \vspace*{-.325\baselineskip}
    \noindent\usebox{\captiontextBox}%
    \vspace*{-.56\baselineskip}%
    \kern 2pt\hrule\kern 2pt\relax
  }

  \def\FrameCommand{
    \hspace{-.007\textwidth}%
    \CustomFBox
      {\fullcaption}
      {\vspace{.13\baselineskip}}
      {.8pt}{.4pt}{0pt}{0pt}
  }

  \def\FirstFrameCommand{
    \hspace{-.007\textwidth}%
    \CustomFBox
      {\fullcaption}
      {\hfill\textit{cont}\enspace$\longrightarrow$}
      {.8pt}{0pt}{0pt}{0pt}
  }

  \def\MidFrameCommand{
    \hspace{-.007\textwidth}%
    \CustomFBox
      {$\longrightarrow$\enspace\textit{cont}\par\vspace*{.3\baselineskip}}
      {\hfill\textit{cont}\enspace$\longrightarrow$}
      {0pt}{0pt}{0pt}{0pt}
  }

  \def\LastFrameCommand{
    \hspace{-.007\textwidth}%
    \CustomFBox
      {$\longrightarrow$\enspace\textit{cont}\par\vspace*{.3\baselineskip}}
      {\vspace{.13\baselineskip}}
      {0pt}{.4pt}{0pt}{0pt}
  }

  \MakeFramed{\FrameRestore}

}{
  \endMakeFramed
}
